\chapter{Resultaten cross platforme mobiele applicatie}
\label{ch:resultatencrossplatformapp}
In dit hoofdstuk worden de resultaten van het onderzoek besproken, waarbij er 2 belangrijke aspecten besproken worden.
Enerzijds wordt de performantie in verband met gegevensgebruik en  snelheid besproken.
Tevens komen ook de tijd die nodig is om de toepassingen te ontwikkelen en de gebruiksvriendelijkheid aan bod in de hoofdstuk.
Deze resultaten dienen overigens als basis om later een besluit onder welke voorwaarden wanneer men best kiest voor de
cross-platforme mobiele applicatie en onder welke voorwaarden men beter kiest voor de responsive website.

\section{Tijd die nodig is om de toepassing te ontwikkelen}
Het eerste gegeven dat in de keuze tussen een cross platforme mobiele applicatie en responsive webapplicatie van belang is,
is de tijd die nodig is om deze toepassing tot stand te brengen. Deze tijd wordt samengerekend met de tijd die nodig is om de
REST-api te ontwikkelen. Dit aangezien de REST dient als basis voor de mobiele applicatie.

De eerste stap, in de ontwikkeling van de cross platforme mobiele applicatie, was het opzetten van REST-api.
Deze REST-api heeft een twee-ledige functie, zoals reeds in het Hoofdstuk "Ontwikkeling cross platforme mobiele applicatie" vermeld staat.
Hierbij werd eerst het beschikbaar maken van de data uit een achterliggende databank gerealiseerd. Nadien is de beveiliging aan de hand
van authenticatie-token toegevoegd. De ontwikkeling van deze REST-api heeft ... dagen in beslag genomen. Hieronder vind u in uitgebreid
overzicht van de tijden die de ontwikkeling van de mobiele applicatie.

%% Tabel met tijden invoegen.








\section{Snelheid van de applicatie}


\section{Gegevensverbruik van de cross platforme mobiele applicatie}

\section{Mogelijke problemen bij de ontwikkeling}
