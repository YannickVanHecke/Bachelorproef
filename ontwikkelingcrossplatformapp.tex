\chapter{Ontwikkeling cross platforme mobiele applicatie}
\label{ch:ontwikkelingcrossplatformapp}
\section{Analyse van het project}
Als start van het project werd er een analyse gemaakt van de co-promotor
een analyse gemaakt van de functionele requirements.

De analyse bestaat uit het sturen van een query naar de databank die draait in een webservice.
Op deze manier worden de gegevens omgezet naar JSON \footnote{JavaScript Object Notation}, een gegevensformaat dat makkelijk
leesbaar is voor toepassingen en vaak gebruikt wordt in webservice.  Het grote voordeel van \cite{inleidingtotjson}
is dat het makkelijk leesbaar/te genereren is voor/door toepassingen.

\section{Ontwikkeling van de ASP.NET MVC Web api}
De ASP.NET MVC Web api heeft in deze toepassing 2 functies. Enerzijds wordt de web api gebruikt om gebruikers zich te laten
authenticeren bij het Active Directory domein van de Gentse Politie, anderzijds levert de web api de gegevens die binnen
de applicatie nodig zijn voor de gebruikers. Deze authenticatie gebeurt aan de hand van tokens die een bepaalde tijd geldig zijn
(in deze casus een week). Nadien moet de gebruiker zich opnieuw authenticeren aan de hand van de inloggegevens. Dit om opnieuw een
token met geldigheid van 1 week te verkrijgen. De tokens zijn verplicht toe te voegen aan de request. In dit niet gebeurt stuurt
de server een antwoord dat men geen toegang heeft tot de aangevraagde resources.

\section{Ontwikkeling van de cross platforme mobiele applicatie}
\subsection{Gedeelde code}
\subsection{Code specifiek voor Android}
\subsection{Code specifiek voor iOS}
\subsection{Code specifiek voor windowsphone}
