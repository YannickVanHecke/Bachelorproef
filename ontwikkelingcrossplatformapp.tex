\chapter{Ontwikkeling cross platforme mobiele applicatie}
\label{ch:ontwikkelingcrossplatformapp}
\section{Analyse van het project}
Als start van het project werd er een analyse gemaakt van de co-promotor
een analyse gemaakt van de functionele requirements.

De analyse bestaat uit het sturen van een query naar de databank die draait in een webservice.
Op deze manier worden de gegevens omgezet naar JSON \footnote{JavaScript Object Notation}, een gegevensformaat dat makkelijk
leesbaar is voor toepassingen en vaak gebruikt wordt in webservice.  Het grote voordeel van \cite{inleidingtotjson}
is dat het makkelijk leesbaar/te genereren is voor/door toepassingen.

\section{Ontwikkeling van de ASP.NET MVC Web api}
De ASP.NET MVC Web api heeft in deze toepassing 2 functies. Enerzijds wordt de web api gebruikt om gebruikers zich te laten
authenticeren bij het Active Directory domein van de Gentse Politie, anderzijds levert de web api de gegevens die binnen
de applicatie nodig zijn voor de gebruikers. Deze authenticatie gebeurt aan de hand van tokens die een bepaalde tijd geldig zijn
(in deze casus een week). Nadien moet de gebruiker zich opnieuw authenticeren aan de hand van de inloggegevens. Dit om opnieuw een
token met geldigheid van 1 week te verkrijgen. De tokens zijn verplicht toe te voegen aan de request. In dit niet gebeurt stuurt
de server een antwoord dat men geen toegang heeft tot de aangevraagde resources.

\section{Ontwikkeling van de cross platforme mobiele applicatie}
\subsection{Gedeelde code}
Om de applicaties op een zo efficiënte manier te ontwikkelen, wordt er code gedeeld tussen de deelprojecten van de 3 mobiele
platformen. Concreet houdt dit in de code die nodig is voor het ophalen van de gegevens uit de webservice gemeenschappelijk
gesteld wordt voor zowel het Android-, als het iOS- en Windowsphone-project. Ook het domeinmodel wordt gemeenschappelijk
gedefinieërd voor de drie mobiele applicatie.

\subsubsection{Code om data op te halen uit de REST-api}
De code die absolute gedeeld word wegens efficientieredenen is de code om de data op te halen uit de REST-service.





\begin{lstlisting}
using System.Threading.Tasks;
using Newtonsoft.Json;
using System.Net.Http;

namespace WeatherApp
{
  public class DataService
  {
    public async Task<dynamic> getData(string queryString)
    {
      HttpClient client = new HttpClient();
      var response = await client.GetAsync(queryString);

      dynamic data = null;
      if (response != null)
      {
        string json = response.Content.ReadAsStringAsync().Result;
        data = JsonConvert.DeserializeObject(json);
      }
      return data;
    }
  }
}

\end{lstlisting}
\begin{lstlisting}
using System;
using System.Threading.Tasks;

namespace WeatherApp
{
  public class Core
  {
    public async Task<Weather> GetWeather(string zipCode)
    {
      string key = "YOUR KEY HERE";
      string queryString = "http://api.openweathermap.org/data/2.5/weather?zip="
                + zipCode + ",us&appid=" + key + "&units=imperial";

      //Make sure developers running this sample replaced the API key
      if (key == "YOUR API KEY HERE")
      {
        throw new ArgumentException("You must obtain an API key " +
                "from openweathermap.org/appid and save it in the 'key' " +
                " variable.");
      }
      dynamic results = await DataService.getDataFromService(queryString)
              .ConfigureAwait(false);

      if (results["weather"] != null)
      {
        Weather weather = new Weather();
        weather.Title = (string)results["name"];
        weather.Temperature = (string)results["main"]["temp"] + " F";
        weather.Wind = (string)results["wind"]["speed"] + " mph";
        weather.Humidity = (string)results["main"]["humidity"] + " %";
        weather.Visibility = (string)results["weather"][0]["main"];

        DateTime time = new System.DateTime(1970, 1, 1, 0, 0, 0, 0);
        DateTime sunrise = time.AddSeconds((double)results["sys"]["sunrise"]);
        DateTime sunset = time.AddSeconds((double)results["sys"]["sunset"]);
        weather.Sunrise = sunrise.ToString() + " UTC";
        weather.Sunset = sunset.ToString() + " UTC";
        return weather;
      }
      else
      {
        return null;
      }
    }
  }
}
\end{lstlisting}

\subsection{Platform specifieke code}
De deelprojecten voor de drie mobiele applicaties bestaat enerzijds uit de definitie van de schermen voor de mobiele applicaties.
Anderzijds is ook de code die de gebruikersevent van de GUI opvangen. In deze code wordt er vervolgens een request gestuurd naar
de api voor authenticatie en data.
