%%========================================================================
%% LaTeX sjabloon voor bachelorproef toegepaste informatica
%%  HoGent Bedrijf en Organisatie - Bijlagenbundel
%%========================================================================

%%========================================================================
%% Preamble
%%========================================================================

\documentclass[pdftex,a4paper,12pt,twoside]{report}

% XXX: Let op: dit sjabloon is gemaakt om dubbelzijdig af te drukken
% Voor enkelzijdig, verwijder ``twoside'' hierboven.

%%---------- Extra functionaliteit ---------------------------------------

\usepackage[utf8]{inputenc}  % Accenten gebruiken in tekst (vb. é ipv \'e)
\usepackage{amsfonts}        % AMS math packages: extra wiskundige
\usepackage{amsmath}         %   symbolen (o.a. getallen-
\usepackage{amssymb}         %   verzamelingen N, R, Z, Q, etc.)
\usepackage[dutch]{babel}    % Taalinstellingen: woordsplitsingen,
                             %  commando's voor speciale karakters
                             %  ("dutch" voor NL)
\usepackage{eurosym}         % Euro-symbool €
\usepackage{geometry}
\usepackage{graphicx}        % Invoegen van tekeningen
\usepackage[pdftex,bookmarks=true]{hyperref}
                             % PDF krijgt klikbare links & verwijzingen,
                             %  inhoudstafel
\usepackage{listings}        % Broncode mooi opmaken
\usepackage{multirow}        % Tekst over verschillende cellen in tabellen
\usepackage{rotating}        % Tabellen en figuren roteren
\usepackage{natbib}          % Betere bibliografiestijlen
\usepackage{fancyhdr}        % Pagina-opmaak met hoofd- en voettekst

\usepackage[T1]{fontenc}     % Ivm lettertypes
\usepackage{lmodern}
\usepackage{textcomp}

\usepackage{lipsum}          % Voor vultekst (lorem ipsum)

%%---------- Layout ------------------------------------------------------

% hoofdingen, enz.
\pagestyle{fancy}
% enkel hoofdstuktitel in hoofding, geen sectietitel (vermijd overlap)
\renewcommand{\sectionmark}[1]{}

% lijn, wordt gebruikt in titelpagina
\newcommand{\HRule}{\rule{\linewidth}{0.5mm}}

% Leeg blad
\newcommand{\emptypage}{
\newpage
\thispagestyle{empty}
\mbox{}
\newpage
}

% Gebruik een schreefloos lettertype ipv het "oubollig" uitziende
% Computer Modern
\renewcommand{\familydefault}{\sfdefault}

% Commando voor invoegen Java-broncodebestanden (dank aan Niels Corneille)
% Gebruik: \javacode{source/MijnKlasse.java}{Uitleg bij de code}
\newcommand{\javacode}[2]{ \lstset{%
  language=java,
  breaklines=true,
  float=th,
  caption={#2},
  basicstyle=\scriptsize,
  frame=single,
  extendedchars=\true
}
\lstinputlisting{#1}}

%%---------- Documenteigenschappen ---------------------------------------
%% Vul dit aan met je eigen info:

% Je eigen naam
\newcommand{\student}{Yannick Van Hecke}

% De naam van je lector, begeleider, promotor
\newcommand{\promotor}{Koen Hoof}

% De naam van je co-promotor
\newcommand{\copromotor}{Jeroen Gevenois}

% Indien je bachelorproef in opdracht van een bedrijf of organisatie
% geschreven is, geef je hier de naam.
\newcommand{\instelling}{Politiezone Gent - Dienst ICT}

% De titel van het rapport/bachelorproef
\newcommand{\titel}{Vergelijkende studie en proof-of-concept in functie van een doelbewuste keuze tussen een cross platform mobiele applicatie en een mobiele website}

% Datum van indienen
\newcommand{\datum}{2 juni 2017}

% Faculteit
\newcommand{\faculteit}{Faculteit Bedrijf en Organisatie}

% Soort rapport
\newcommand{\rapporttype}{Scriptie voorgedragen tot het bekomen van de graad van\\Bachelor in de toegepaste informatica}

% Academiejaar
\newcommand{\academiejaar}{2016-2017}

% Examenperiode
%  - 1e semester = 1e examenperiode
%  - 2e semester = 2e examenperiode
%  - tweede zit = 3e examenperiode
\newcommand{\examenperiode}{Tweede examenperiode}

%%========================================================================
%% Inhoud document
%%========================================================================

\begin{document}

%%---------- Front matter ------------------------------------------------
%% Het voorblad - Hier moet je in principe niets wijzigen.

\begin{titlepage}
  \newgeometry{top=2cm,bottom=1.5cm,left=1.5cm,right=1.5cm}
  \begin{center}

    \begingroup
    \rmfamily
    \includegraphics[width=2.5cm]{img/HG-beeldmerk-woordmerk}\\[.5cm]
    \faculteit\\[3cm]
    \titel\\[1cm]
    Bijlagen
    \vfill
    \student\\[3.5cm]
    \rapporttype\\[2cm]
    Promotor:\\
    \promotor\\
    Co-promotor:\\
    \copromotor\\[2.5cm]
    Instelling: \instelling\\[.5cm]
    Academiejaar: \academiejaar\\[.5cm]
    \examenperiode
    \endgroup

  \end{center}
  \restoregeometry
\end{titlepage}

% Schutblad

\emptypage


\begin{titlepage}
  \newgeometry{top=5.35cm,bottom=1.5cm,left=1.5cm,right=1.5cm}
  \begin{center}

    \begingroup
    \rmfamily
    \faculteit\\[3cm]
    \titel\\[1cm]
    Bijlagen
    \vfill
    \student\\[3.5cm]
    \rapporttype\\[2cm]
    Promotor:\\
    \promotor\\
    Co-promotor:\\
    \copromotor\\[2.5cm]
    Instelling: \instelling\\[.5cm]
    Academiejaar: \academiejaar\\[.5cm]
    \examenperiode
    \endgroup

  \end{center}
  \restoregeometry
\end{titlepage}


\tableofcontents

\appendix

\chapter{Uitgebreide testresultaten met betrekking tot tijden}
\label{ch:uitgebereidresultatenmetbetrekkingtottijden}

% Automatisch invoegen van al je Java broncode:
% % 1/ maak een link naar je broncodedirectory naar subdirectory source
% %      ln -s /path/to/java/src/ ./source
% %    Of kopieer desnoods al je broncodebestanden. Zorg dat je
% %    versiebeheersysteem deze directory negeert!
% % 2/ Genereer source.tex met het script source.sh
% %      ./source.sh
% % 3/ Haal volgende regel uit commentaar
% %\input{source.tex}
\subsection{Resultaten voor het opstarten}
Alle tijden die vermeld worden, werden gemeten in milliseconden.
\subsubsection{Resultaten voor Android}
\begin{center}
    \begin{tabular}{ | l | l | l |}
    \hline
    Poging & Android app & webapp op Android
      \\ \hline
      1 & 769,8043 & 820

      \\ \hline
      2 & 801,6959 & 639

      \\ \hline
      3 & 823,9043 & 688

      \\ \hline
      4 & 882,1604 & 612

      \\ \hline
      5 & 878,9123 & 701

      \\ \hline
      6 & 873,6580 & 901

      \\ \hline
      7 & 850,0426 & 665

      \\ \hline
      8 & 770,9071 & 844

      \\ \hline
      9 & 861,8291 & 725

      \\ \hline
      10 & 804,0946 & 662

      \\ \hline
      11 & 812,6415 & 674

      \\ \hline
      12 & 791,5941 & 805

      \\ \hline
      13 & 777,4317 & 616

      \\ \hline
      14 & 811,9849 & 624

      \\ \hline
      15 & 853,5280 & 602

      \\ \hline
      16 & 833,6636 & 576

      \\ \hline
      17 & 887,3326 & 603

      \\ \hline
      18 & 905,7289 & 722

      \\ \hline
      19 & 818,4091 & 630

      \\ \hline
      20 & 878,5551 & 699
    \end{tabular}
\end{center}

\subsection{Uitgebereide resultaten voor windowsphone}
\chapter{Uitgebereide testresultaten voor de responsive webapplicatie}
\subsection{Uitgebreide resultaten voor Android}
\subsection{Uitgebereide resultaten voor windowsphone}
\end{document}
