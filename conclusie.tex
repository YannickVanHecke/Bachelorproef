%%=============================================================================
%% Conclusie
%%=============================================================================

\chapter{Conclusie}
\label{ch:conclusie}

%% TODO: Trek een duidelijke conclusie, in de vorm van een antwoord op de
%% onderzoeksvra(a)g(en). Wat was jouw bijdrage aan het onderzoeksdomein en
%% hoe biedt dit meerwaarde aan het vakgebied/doelgroep? Reflecteer kritisch
%% over het resultaat. Had je deze uitkomst verwacht? Zijn er zaken die nog
%% niet duidelijk zijn? Heeft het ondezoek geleid tot nieuwe vragen die
%% uitnodigen tot verder onderzoek?

%%\lipsum[76-80]

%% Snelst resultaat -> website -> beter kennis

%% efficiënst data -> app -> kan nog verder geoptimaliseerd worden mits caching van data

%% snelst gevraagde gegevens -> app -> invloed van caching moet verder ontzocht worden

%% Vermelden dat alle resultaten zonder caching zijn gemeten en dit in de
%% toekomst verder onderzocht kan/moet worden om een preciezere conclusie
%% te kunnen trekken. Hiervoor dient wel de Web api aangepast te worden zodat
%% men alle gegevens na een bepaalde persoon kan opvragen.

%% ook vermelden dat voor het iOS project enkel het inloggen gerealiseerd is.

In dit werk werd een vergelijkende studie en proof-of-concept in functie van een doelbewuste keuze tussen een
cross platforme mobiele applicatie en een responsive website opgezet. Hierbij werden verschillende aspecten van de ontwikkeling
en de toepassing onderzocht. Hierbij kwamen volgende onderzoeksvragen aan bod:

\begin{itemize}
  \item{Bij welke vorm van applicatie wordt het snelst resultaat behaald?}
  \item{Welke optie is het efficiënst?}
  \begin{itemize}
    \item{Welke optie verbruikt de meeste hoeveelheid gegevens?}
    \item{Welke optie geeft de eindgebruiker het snelst de gevraagd informatie?}
  \end{itemize}
  \item{Welke optie kan het snelst ontwikkeld worden?}
\end{itemize}

Uit het gevoerde onderzoek resulteerde dat de mobiele applicatie op de eerste onderzoeksvraag, een beter resultaat haalt dan de responsive
webapplicatie. Dit wegens de betere kennis van responsive webapplicatie. Mogelijks verschillend dit van ontwikkelaar tot ontwikkelaar,
afhankelijk van de ervaring met beide technologieën. Verder dient ook het feit dat bij de mobiele applicatie men voor elk mobiel besturingssysteem
een afgeslankte mobiele toepassing te ontwikkelen. Dit doet de ontwikkeltijd oplopen in vergelijking met de responsive website.

Over de efficientie van de mobiele applicatie en de responsive webapplicatie is er wel een duidelijk voordeel voor de mobiele applicatie.
Dit komt omdat het gegevensverbruik en de snelheid bij de mobiele applicatie lager ligt. Dit is te verklaren omdat er bij de mobiele applicatie
enkel de weer te geven data moet gedownload worden, in tegenstelling tot de responsive website, waarbij er naast de weer te geven data, ook nog html
en css gedownload moet worden om de data op een geschikte manier weer te geven.
