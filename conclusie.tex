%%=============================================================================
%% Conclusie
%%=============================================================================

\chapter{Conclusie}
\label{ch:conclusie}

%% TODO: Trek een duidelijke conclusie, in de vorm van een antwoord op de
%% onderzoeksvra(a)g(en). Wat was jouw bijdrage aan het onderzoeksdomein en
%% hoe biedt dit meerwaarde aan het vakgebied/doelgroep? Reflecteer kritisch
%% over het resultaat. Had je deze uitkomst verwacht? Zijn er zaken die nog
%% niet duidelijk zijn? Heeft het ondezoek geleid tot nieuwe vragen die
%% uitnodigen tot verder onderzoek?

%%\lipsum[76-80]

%% Snelst resultaat -> website -> beter kennis

%% efficiënst data -> app -> kan nog verder geoptimaliseerd worden mits caching van data

%% snelst gevraagde gegevens -> app -> invloed van caching moet verder ontzocht worden

%% Vermelden dat alle resultaten zonder caching zijn gemeten en dit in de
%% toekomst verder onderzocht kan/moet worden om een preciezere conclusie
%% te kunnen trekken. Hiervoor dient wel de Web api aangepast te worden zodat
%% men alle gegevens na een bepaalde persoon kan opvragen.

%% ook vermelden dat voor het iOS project enkel het inloggen gerealiseerd is.

In dit werk werd een vergelijkende studie en proof-of-concept in functie van een doelbewuste keuze tussen een
cross platform mobiele applicatie en een responsive website opgezet.

Hierbij werden verschillende aspecten van de ontwikkeling
en de toepassing onderzocht. Hierbij kwamen volgende onderzoeksvragen aan bod:

\begin{itemize}
  \item{Bij welke vorm van applicatie wordt het snelst resultaat behaald?}
  \item{Welke optie is het efficiënst?}
  \begin{itemize}
    \item{Welke applicatie verbruikt de meeste hoeveelheid gegevens?}
    \item{Welke optie geeft de eindgebruiker het snelst de gevraagde informatie?}
  \end{itemize}
\end{itemize}

Het gevoerde onderzoek leerde dat de responsive webapplicatie op de eerste onderzoeksvraag, sneller resultaat haalt dan de mobiele applicatie.

Dit wegens de betere kennis van responsive webapplicatie. Mogelijks verschilt dit van ontwikkelaar tot ontwikkelaar,
afhankelijk van de ervaring met beide technologieën.

Verder moet men ook in overweging nemen dat bij de ontwikkeling van de mobiele applicatie, men voor elk mobiel besturingssysteem een afgeslankte mobiele toepassing dient te ontwikkelen.
Dit doet de ontwikkeltijd oplopen in vergelijking met de responsive website.

Over de efficientie van de mobiele applicatie en de responsive webapplicatie is er wel een duidelijk voordeel voor de mobiele applicatie.
De reden hiervoor is dat het gegevensverbruik lager is bij de responsive website.
Bij de tijden valt echter op dat deze bij Android hoger zijn op de mobiele applicatie. Dit in tegenstelling tot windowsphone, waar de tijden voor de mobiele applicatie onder die van de responsive website liggen.

Dit is te verklaren omdat er bij de mobiele applicatie op windowsphone enkel de weer te geven data moet gedownload worden. Dit in tegenstelling tot de responsive website, waarbij er naast de weer te geven data, ook nog html
en css gedownload moet worden om de data op een geschikte manier te tonen. Een verklaring voor de hoger tijden op Android bij de mobiele applicatie werd er niet gevonden.

Het resultaat van de mobiele applicatie kan mogelijks nog verder geoptimaliseerd worden vermits in deze casus het effect van cachen of lokaal opslaan van gegevens op het resultaat niet onderzocht is.
Om het lokaal opslaan van data optimaal te laten functioneren, dient de REST api aangepast te worden.

Op de vraag welke optie de eindgebruiker het snelst de gevraagde gegevens geeft, blijkt er geen eenduidige voorkeur uit het onderzoek voort te vloeien.
Dit omdat uit de testresultaten bij Android blijkt dat de respsonive webapplicatie hier beter functioneert. Dit in tegenstelling tot windowsphone, waar de mobiele applicatie betere
resultaten behaalt.

Verder dient men ook rekening te houden dat er voor iOS slechts een minimale applicatie ontwikkeld werd.

Om in de toekomst een meer accuraat antwoord te geven op deze onderzoeksvragen, moet het caching van data in de mobiele applicatie voorzien worden.
Dit is momenteel nog niet geïmplementeerd wegens tijdsgebrek.
