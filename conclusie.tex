%%=============================================================================
%% Conclusie
%%=============================================================================

\chapter{Conclusie}
\label{ch:conclusie}

%% TODO: Trek een duidelijke conclusie, in de vorm van een antwoord op de
%% onderzoeksvra(a)g(en). Wat was jouw bijdrage aan het onderzoeksdomein en
%% hoe biedt dit meerwaarde aan het vakgebied/doelgroep? Reflecteer kritisch
%% over het resultaat. Had je deze uitkomst verwacht? Zijn er zaken die nog
%% niet duidelijk zijn? Heeft het ondezoek geleid tot nieuwe vragen die
%% uitnodigen tot verder onderzoek?

%%\lipsum[76-80]

%% Snelst resultaat -> website -> beter kennis

%% efficiënst data -> app -> kan nog verder geoptimaliseerd worden mits caching van data

%% snelst gevraagde gegevens -> app -> invloed van caching moet verder ontzocht worden

%% Vermelden dat alle resultaten zonder caching zijn gemeten en dit in de
%% toekomst verder onderzocht kan/moet worden om een preciezere conclusie
%% te kunnen trekken. Hiervoor dient wel de Web api aangepast te worden zodat
%% men alle gegevens na een bepaalde persoon kan opvragen.

%% ook vermelden dat voor het iOS project enkel het inloggen gerealiseerd is.

In dit werk werd een vergelijkende studie en proof-of-concept in functie van een doelbewuste keuze tussen een
cross platforme mobiele applicatie en een responsive website opgezet. Hierbij werden verschillende
