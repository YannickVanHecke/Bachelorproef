
%%=============================================================================
%% Inleiding
%%=============================================================================

\chapter{Inleiding}
\label{ch:inleiding}

%% De inleiding moet de lezer alle nodige informatie verschaffen om het onderwerp te begrijpen zonder nog externe werken te moeten raadplegen \citep{Pollefliet2011}. Dit is een doorlopende tekst die gebaseerd is op al wat je over het onderwerp gelezen hebt (literatuuronderzoek).

%% Je verwijst bij elke bewering die je doet, vakterm die je introduceert, enz. naar je bronnen.
%% In \LaTeX{} kan dat met het commando \texttt{$\backslash${cite\{\}}} of \texttt{$\backslash${citep\{\}}}.
%%Als argument van het commando geef je de ``sleutel'' van een ``record'' in een bibliografische databank in het Bib\TeX{}-formaat
%% (een tekstbestand). Als je expliciet naar de auteur verwijst in de zin, gebruik je \texttt{$\backslash${}cite\{\}}.
%% Soms wil je de auteur niet expliciet vernoemen, dan gebruik je \texttt{$\backslash${}citep\{\}}.
%%Hieronder een voorbeeld van elk.

\section{Stand van zaken}
\label{sec:stand-van-zaken}

%% TODO: deze sectie (die je kan opsplitsen in verschillende secties) bevat je
%% literatuurstudie. Vergeet niet telkens je bronnen te vermelden!
In de vakliteratuur zijn er vandaag diverse artikelen die een vergelijkende studie maken tussen mobiele applicaties
en responsive webapplicaties. De meeste van deze onderzoeken resulteren vaak in een beperkte lijst van kenmerken die aangeven wanneer men het best voor welke optie kiest.
Dit zonder dat hier wetenschappelijk onderzoek aan vooraf ging. Toch zijn deze wetenschappelijke artikelen op het internet beschikbaar. Zo onderzocht \cite{albuquerque2015}
in het artikel 'CROSS PLATFORM APP A COMPARATIVE STUDY' de voornaamste reden om te kiezen voor enerzijds de mobiele
applicatie en anderzijds de responsive website.

De belangrijkste redenen uit het onderzoek om te kiezen voor een cross-platform mobiele applicatie zijn de volgende:
\begin{itemize}
  \item{performantie}
  \item{user interfaces in de lijn van het besturingssysteem}
  \item{distributie via de app stores}
  \item{push notificaties}
\end{itemize}

In de vakliteratuur haalt men de volgende redenen om te kiezen voor een responsive webapplicatie aan:
\begin{itemize}
  \item{eenvoudige ondersteuning voor meerdere systemen}
  \item{het publiceren van de toepassing}
\end{itemize}



\section{Probleemstelling en Onderzoeksvragen}
\label{sec:onderzoeksvragen}

%% TODO:
%% Uit je probleemstelling moet duidelijk zijn dat je onderzoek een meerwaarde
%% heeft voor een concrete doelgroep (bv. een bedrijf).
%%
%% Wees zo concreet mogelijk bij het formuleren van je
%% onderzoeksvra(a)g(en). Een onderzoeksvraag is trouwens iets waar nog
%% niemand op dit moment een antwoord heeft (voor zover je kan nagaan).

Momenteel wordt er bij de ICT-afdeling van de Gentse Politie aan verschillende projecten gewerkt waarbij het
mobiel beschikbaar maken van interne gegevens een vereiste is.
Om dit naar een gebruiksklaar eindproduct voor een mobiel apparaat om te zetten, bestaan er vandaag de twee opties.

Enerzijds kan men kiezen voor een cross-platform mobiele applicatie.
Dit houdt in dat men binnen 1 ontwikkelomgeving een applicatie ontwikkelt die bruikbaar is op de
3 grote mobiele platformen. Deze zijn: Android van Google, iOS van Apple en Windows Phone of
Windows Universal Platform van Microsoft.
Hierbij wordt er code gedeeld tussen de 3 platformen.

Het weergeven van de data gebeurt op een platform-specifieke manier.
Dit omdat men binnen de verschillende mobiele besturingssystemen werkt met andere UI elementen om de data in te tonen.

Anderzijds bestaat de mogelijkheid om te kiezen voor een responsive webapplicatie.
Dit wil zeggen dat de inhoud van de webapplicatie aangepast wordt,
 naargelang het scherm waarbij men de mobiele website bekijkt.
Zo wordt er op een smartphone mogelijks minder informatie weergegeven dan indien men de website bekijk op een computer of
wordt in het geval van de smartphone de positie en afmetingen van de elementen aangepast.

In dit werk wordt onderzocht in welke gevallen de webapplicatie de beste optie is alsook in welke gevallen de mobiele applicatie
een beter alternatief vormt.

Hierbij zullen volgende onderzoeksvragen beantwoord worden:
\begin{itemize}
  \item{Bij welke vorm van applicatie biedt het snelst resultaat?}
  \item{Welke keuze is het efficiënst?}
  \begin{itemize}
    \item{Welke vorm van applicatie verbruikt de meeste hoeveelheid gegevens?}
    \item{Welke optie geeft de eindgebruiker het snelst de gevraagde informatie?}
  \end{itemize}
\end{itemize}
\newpage
\section{Opzet van deze bachelorproef}
\label{sec:opzet-bachelorproef}
In functie van de snelheid van de ontwikkeling van beide producten, wordt er een ontwikkelomgeving gekozen.
Om een cross-platform mobiele applicatie te ontwikkelen, bestaat de keuze uit 2 technologieën . Enerzijds zijn er
ontwikkelomgevingen met C\# als achterliggende programmeertaal. Hierbij wordt de UI gedefinieërd in xml voor Android,
xaml van Windows Phone en Universal Windows Platform en storyboards voor iOS. Hierbij wordt de oorspronkelijke interactie
en structuur van de UI en de logica van het mobiele besturingssysteem behouden. Visual Studio en Xamarin zijn 2 voorbeelden
die gebaseerd zijn op deze technologie. Anderzijds kan men ook opteren voor een cross-platforme mobiele applicatie te bouwen op basis van html, css en javascript.
Hierbij wordt er eerst een webservice van het te hosten project gemaakt, om vervolgens deze webservice op te starten in een emulator.
De ontwikkelomgeving PhoneGap is op deze manier van werken gebaseerd \citep{adobesystemsinc2017}.

In dit onderzoek zal gekozen worden voor de eerste manier van werken, namelijk de ontwikkelomgeving met C\# als achterliggende
programmeertaal. Dit vanwege de huidige kennis van het ontwikkelen van cross-platforme mobiele applicatie.

Nadat de gekozen ontwikkelingomgeving vastligt, wordt er aan de hand van de functionele requirements beslist wat de
minimale versies van de besturingssystemen zijn. Hierbij wordt getracht het aantal ondersteunende apparaten zo hoog mogelijk te houden.
Tijdens het onderzoek wordt, zoals in de onderzoeksvragen reeds vermeld, ook rekening gehouden met de tijd en kosten die de ontwikkeling
en publiek stellen van beide applicaties met zich meebrengen.

Vervolgens wordt de gebruikte methodologie toegelicht, waarna de casus uitgelegd wordt.
Deze casus vormt de basis in het onderzoek naar een efficiënte keuze tussen de
cross-platforme mobiele applicatie en de responsive webapplicatie.

Nadat de casus duidelijk uiteengezet is, worden de gemaakte applicaties verduidelijkt aan de hand van de geschreven code.
Vervolgens worden de resultaten getoond en wordt de conclusie, die uit het onderzoek voortvloeit, medegedeeld.
Deze conclusie wordt ook gebruikt door de Gentse politie in hun keuze tussen een mobiele applicatie of een responsive website.

Na de conclusie worden er nog richtlijnen voor toekomstig onderzoek gegeven in deze materie.

%% TODO: Het is gebruikelijk aan het einde van de inleiding een overzicht te
%% geven van de opbouw van de rest van de tekst. Deze sectie bevat al een aanzet
%% die je kan aanvullen/aanpassen in functie van je eigen tekst.
