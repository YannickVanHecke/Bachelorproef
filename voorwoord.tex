%%=============================================================================
%% Voorwoord
%%=============================================================================

\chapter*{Voorwoord}
\label{ch:voorwoord}

%% TODO:
%% Het voorwoord is het enige deel van de bachelorproef waar je vanuit je
%% eigen standpunt (``ik-vorm'') mag schrijven. Je kan hier bv. motiveren
%% waarom jij het onderwerp wil bespreken.
%% Vergeet ook niet te bedanken wie je geholpen/gesteund/... heeft

In deze bachelorproef worden de verschillen in performantie tussen een cross-platforme mobiele applicatie en responsive website onderzocht.
De keuze voor dit onderwerp is gekomen na een opdrachtswijziging in de stage van het vorige academiejaar 2015-2016. Hierbij werd er na
een kosten-analyse van een cross-platforme mobiele applicatie beslist om over te schakelen op een responsive website. Toen werd de
interesse om dit uitgebereider te onderzoek, getriggerd.

Deze vergelijkende studie en proof-of-concept in functie van een doelbewuste keuze tussen een cross platform mobiele applicatie en
een mobiele website werd gevoerd tussen februari 2017 en mei 2017 bij de ICT-afdeling van de Gentse Politie, waarvoor mijn dank aan mijn
co-promotor Jeroen Gevenois voor bij te staan in mijn onderzoek, met antwoorden op mijn vragen tijdens het ontwikkelen van beide toepassingen.

Verder wil ik ook Matthias Vercaemst bedanken om mij te helpen met de verschillende netwerken-vragen die ik had tijdens het onderzoek.

Ook Diensthoofd Ruben Vansevenant zou ik graag willen bedanken om mij de kans te geven om het onderwerp van mijn bachelorproef uit te werken.

Verder wil ik ook mijn promotor Koen Hoof bedanken voor de feedback en de opvolging van deze bachelorproef tijdens de afgelopen maanden.
Deze feedback hielp mij om telkens dit werk te verbeteren.

Tot slot bedank ik nog mijn Gon-Begeleider Elke Van Paemel voor de steun tijdens het schrijven van de bachelorproef.
