%%=============================================================================
%% Voorwoord
%%=============================================================================

\chapter*{Voorwoord}
\label{ch:voorwoord}

%% TODO:
%% Het voorwoord is het enige deel van de bachelorproef waar je vanuit je
%% eigen standpunt (``ik-vorm'') mag schrijven. Je kan hier bv. motiveren
%% waarom jij het onderwerp wil bespreken.
%% Vergeet ook niet te bedanken wie je geholpen/gesteund/... heeft

In deze bachelorproef worden de verschillen in performantie tussen een cross-platforme mobiele applicatie en responsive website onderzocht.
De keuze voor dit onderwerp is gekomen na een opdrachtswijziging in de stage van het vorige academiejaar (2015-2016).

Toen werd er na een kostenanalyse van een cross-platforme mobiele applicatie beslist om over te schakelen op een responsive website.
Het was dan dat mijn interesse om dit uitgebreider te onderzoeken, getriggerd werd.

Deze vergelijkende studie en proof-of-concept in functie van een doelbewuste keuze tussen een cross platforme mobiele applicatie en
een mobiele website werd gevoerd tussen februari en mei 2017 bij de ICT-afdeling van de Gentse Politie.

Daarom wens ik ook mijn dank te betuigen aan mijn co-promotor Jeroen Gevenois om mij steeds bij te staan tijden mijn onderzoeken en voor
het beantwoorden van mijn vragen tijdens het ontwikkelen van beide toepassingen.

Ook wil ik ook Matthias Vercaemst bedanken om mij te helpen met de verschillende netwerken-vragen die ik had tijdens het onderzoek.

Ook diensthoofd Ruben Vansevenant zou ik graag willen vermelden in mijn bedankingen omdat hij mij de kans te geven om deze bachelorproef uit te werken.

Verder wil ik ook mijn promotor Koen Hoof bedanken voor de steeds opbouwende feedback en de stipte opvolging van deze bachelorproef tijdens de afgelopen maanden.
Deze feedback hielp mij om telkens weer dit werk te verbeteren.

Tot slot bedank ik nog mijn Gon-Begeleider Elke Van Paemel voor de steun tijdens het schrijven van de bachelorproef.
