%%=============================================================================
%% Samenvatting
%%=============================================================================

%% TODO: De "abstract" of samenvatting is een kernachtige (~ 1 blz. voor een
%% thesis) synthese van het document.
%%
%% Deze aspecten moeten zeker aan bod komen:
%% - Context: waarom is dit werk belangrijk?
%% - Nood: waarom moest dit onderzocht worden?
%% - Taak: wat heb je precies gedaan?
%% - Object: wat staat in dit document geschreven?
%% - Resultaat: wat was het resultaat?
%% - Conclusie: wat is/zijn de belangrijkste conclusie(s)?
%% - Perspectief: blijven er nog vragen open die in de toekomst nog kunnen
%%    onderzocht worden? Wat is een mogelijk vervolg voor jouw onderzoek?
%%
%% LET OP! Een samenvatting is GEEN voorwoord!

%%---------- Nederlandse samenvatting -----------------------------------------
%%
%% TODO: Als je je bachelorproef in het Engels schrijft, moet je eerst een
%% Nederlandse samenvatting invoegen. Haal daarvoor onderstaande code uit
%% commentaar.
%% Wie zijn bachelorproef in het Nederlands schrijft, kan dit negeren en heel
%% deze sectie verwijderen.


%%---------- Samenvatting -----------------------------------------------------
%%
%% De samenvatting in de hoofdtaal van het document

\begin{abstract}
  %%\lipsum[1-4]

  Deze bachelorproef heeft als doel de ontwikkelaar op basis van onderzoek naar ontwikkeltijd, kosten, snelheid en gegevensverbruik
  te helpen in de keuze tussen een cross platforme mobiele applicatie en responsive website. Dit omdat er vandaag de dag weinig artikels
  ter beschikking zijn waarin een gefundeerde keuze word gemaakt voor 1 van beide vormen van toepassingen.

  Dit werk is tot stand gekomen in verschillende fases. Zo werd er eerst een literatuuronderzoek verricht naar eerdere vergelijkende
  studie tussen cross platforme mobiele apps en responsive websites. Na het vastleggen van de casus werd er gekozen voor Visual Studio
  als ontwikkelingomgeving, waarna de ontwikkeling van beide toepassingen volgde. Eens de ontwikkeling afgerond was, werden de tijden voor
  opstarten, inloggen en gegevens ophalen gemeten samen met de hoeveelheid gegevens die dit met zich meebrengt. Dit om nadien een
  gefundeerde conclusie te vormen op basis van snelheid, gegevensverbruik, kosten. Dit alles zonder caching van data in de mobiele applicatie.

  


\end{abstract}
