%%=============================================================================
%% Samenvatting
%%=============================================================================

%% TODO: De ''abstract'' of samenvatting is een kernachtige (~ 1 blz. voor een
%% thesis) synthese van het document.
%%
%% Deze aspecten moeten zeker aan bod komen:
%% - Context: waarom is dit werk belangrijk?
%% - Nood: waarom moest dit onderzocht worden?
%% - Taak: wat heb je precies gedaan?
%% - Object: wat staat in dit document geschreven?
%% - Resultaat: wat was het resultaat?
%% - Conclusie: wat is/zijn de belangrijkste conclusie(s)?
%% - Perspectief: blijven er nog vragen open die in de toekomst nog kunnen
%%    onderzocht worden? Wat is een mogelijk vervolg voor jouw onderzoek?
%%
%% LET OP! Een samenvatting is GEEN voorwoord!

%%---------- Nederlandse samenvatting -----------------------------------------
%%
%% TODO: Als je je bachelorproef in het Engels schrijft, moet je eerst een
%% Nederlandse samenvatting invoegen. Haal daarvoor onderstaande code uit
%% commentaar.
%% Wie zijn bachelorproef in het Nederlands schrijft, kan dit negeren en heel
%% deze sectie verwijderen.


%%---------- Samenvatting -----------------------------------------------------
%%
%% De samenvatting in de hoofdtaal van het document

\begin{abstract}
  %%\lipsum[1-4]

  Deze bachelorproef heeft als doel de ontwikkelaar op basis van onderzoek naar ontwikkeltijd, kosten, snelheid en gegevensverbruik
  te helpen in de keuze tussen een cross platform mobiele applicatie en responsive website.

  Dit omdat er vandaag de dag weinig artikels
  ter beschikking zijn waarin een gefundeerde keuze wordt gemaakt voor één van beide vormen.

  Deze bachelorproef is tot stand gekomen in verschillende fases. Zo werd er eerst een literatuuronderzoek verricht naar eerdere vergelijkende
  studies tussen beide mogelijkheden.

  Na het vastleggen van de casus werd er gekozen voor Visual Studio als ontwikkelingomgeving. Daarna volgde de ontwikkeling van beide toepassingen.

  Eens de ontwikkeling afgerond was, werden de tijden voor opstarten, inloggen en gegevens ophalen gemeten samen met de hoeveelheid gegevens die dit met zich meebracht.

  Dit alles om, na het analyseren van de verkregen data, een gefundeerde conclusie te vormen op basis van snelheid, gegevensverbruik en kosten. Alle metingen in dit onderzoek gebeurden zonder caching van data in de mobiele applicatie.

  Uit de resultaten van het onderzoek kwam naar voor dat indien men het gegevensverbruik van de toepassing een belangrijk kenmerk vindt, men
  beter kiest voor de mobiele applicatie.

  Wanneer men de snelheid van de toepassing prioritair stelt, heeft de responsive website van Android een duidelijk voordeel ten opzicht van de mobiele applicatie. Bij Windows Phone was mobiele applicatie sneller.
  In het iOS-project is wegens tijdgebrek enkel het inloggen gedeeltelijk geïmplementeerd. Hierop werden geen testen uitgevoerd.

  Verder moeten ook de kosten om beide toepassingen te hosten in rekening genomen. Voor de mobiele applicatie komt dit voor de ASP.NET Web API \footnote{Appliciation Programming Interface} € 92,56 per maand.
  Hierbij moeten nog de kosten voor publicatie in de applicatiewinkel meegerekend worden. Wat Android betreft, is dit eenmalig 25 dollar. Voor Windows Phone is
  dit eenmalig € 16,98 (19 dollar) voor een individueel account en eenmalig € 88,50 (99 dollar) voor een bedrijfsaccount. De kost voor de iOS-applicatie te publiceren komt op 99 dollar per jaar.
  Dit naast de kosten om een macOS-toestel om de iOS-applicatie op te testen.
  Indien deze kosten van belang zijn, kiest men beter voor de responsive webapplicatie. Dit omdat hier enkel de € 92,56 per maand erbij komt.

  De snelheid van ontwikkeling is een criterium dat ook in rekening werd gebracht in deze vergelijkende studie.
  Hierbij kunnen we besluiten dat men bij de ontwikkeling van een responsive website sneller tot een gebruiksklaar eindproduct kon komen.
  De ontwikkeling van de mobiele applicatie had het nadeel dat men drie afzonderlijke deelprojecten moest maken. De uitgebreide resultaten en conclusies
  worden meer in detail toegelicht in de scriptie.



\end{abstract}
