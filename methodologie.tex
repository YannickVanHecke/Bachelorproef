%%=============================================================================
%% Methodologie
%%=============================================================================

\chapter{Methodologie}
\label{ch:methodologie}

%% TODO: Hoe ben je te werk gegaan? Verdeel je onderzoek in grote fasen, en
%% licht in elke fase toe welke stappen je gevolgd hebt. Verantwoord waarom je
%% op deze manier te werk gegaan bent. Je moet kunnen aantonen dat je de best
%% mogelijke manier toegepast hebt om een antwoord te vinden op de
%% onderzoeksvraag.

\section{Literatuuronderzoek}
Ter voorbereiding van deze vergelijkende studie werd er een literatuuronderzoek verricht.
Dit hield in dat er op internet gezocht werd naar verschillende wetenschappelijke artikelen over voorgaande vergelijkende studies
tussen cross platforme mobiele applicatie en responsive website. Dit literatuuronderzoek heeft al voornaamste reden het geven van
een stand van zaken in het ontwikkelen van cross-platforme mobiele applicaties en responsive webapplicatie.

\section{Vastleggen casus}
De eerste stap in het onderzoek naar de vergelijkende studie tussen een cross-platforme mobiele applicatie en een
responsive website, was het vastleggen van de de casus. Dit bepaalde de functionele requirements van beide applicaties en
gebeurde in overleg met de co-promotor. De casus omvatte een toepassing rond personeelsgegevens binnen de gentse politie,
waarbij de gegevens momenteel enkel intern beschikbaar zijn. Het doel van de casus is om deze, mits de nodige beveiliging, beschikbaar
te stellen van het personeel dat niet altijd met het intern netwerk verbonden is. Hierbij word er vooral gedacht aan mensen op het terrein.

\section{Framework kiezen voor cross platform mobiele applicatie}
Omdat vanuit de stage reeds gekozen werd voor de ontwikkelomgeving van Microsoft en dit hierdoor reeds een vertrouwde
manier van werken was, heb ik gekozen om de cross platform mobiele applicatie te ontwikkelen met Visual Studio 2015.
Hierbij wordt er gebruik gemaakt van een apart deelproject binnen de applicatie waarin code gemeenschappelijk wordt gemaakt voor
de 3 platform-specifieke applicaties. Op deze manier kan men code hergebruiken, waardoor de efficiëntie van de geschreven code stijgt.

\section{Framework kiezen voor responsive webapplicatie}
Het framework, dat voor de responsive webapplicatie gebruikt werd, is het MVC-framework van Microsoft.
Hierdoor kan de ontwikkeling van de webapplicatie ook in Visual Studio gebeuren, hetgeen voor de vergelijking van de ontwikkeltijd
een eerlijker resultaat opleverde, dan wanneer men zich nog dient in te werken in nieuwe materie.

\section{Opzetten REST api}
Het opzetten van de REST api was de volgende stap in het onderzoeksproces. Aanvankelijk werd er gekozen om eerst de basisfunctionaliteiten te implementeren en
pas nadien werd de authenticatie toegevoegd. Dit zorgt ervoor dat de data niet ongeoorloofd beschikbaar is voor derden.
Hierbij dient men zich eerst te authenticeren aan de hand van de gebruikersnaam en het wachtwoord van het gebruikersaccount binnen de Politiezone Gent.
Indien men zich succesvol geauthenticeerd heeft bij het domein aan de hand van de Active Directory, krijgt men van de REST api een token terug. De geldigheid van deze token kan men vrij kiezen.
Deze token is verplicht bij te voegen bij alle requests die verstuurd worden naar de server.
Het testen van de REST api zal eerst gebeuren aan POSTMAN, een REST-client. Zo is men zeker dat de REST-api correct werkt, alvorens over te gaan te het ontwikkelen van de mobiele applicatie.

\section{De ontwikkeling van een cross platforme mobiele applicatie}
Nadat het testen van de REST api voltooid is, begint de ontwikkeling van de cross platforme applicatie.
Hierbij is het de bedoeling om de code die gemeenschappelijk is voor de 3 deelprojecten,
nl. het ophalen van gegevens uit de webservice en deze op te slaan in een lokale databank, alsook het mappen naar objecten van deze gegevens
 af te zonderen van de platform-specifieke deelprojecten in een gemeenschappelijk deelproject.
 Op deze manier wordt de ontwikkeltijd gereduceerd ten opzichte van het ontwikkelen van 3 afzonderlijke applicaties.
Het weergeven van de opgehaalde data wordt, samen met de ontwerp van de schermen voor de platforme-specifiek user interfaces,
in de platform-specifieke projecten gecodeerd. Dit omdat er geen uniforme manier bestaat om de user interfaces te voorzien binnen
de gekozen ontwikkelomgeving.

\section{Ontwikkelen responsive webapplicatie}
De volgende stap in deze vergelijkende studie tussen een cross platforme mobiele applicatie en responsive webapplicatie is het
ontwikkelen van de responsive webapplicatie. Hierbij wordt dezelfde databank gebruikt die de webservice van de cross platforme
mobiele applicatie. Verder wordt ook het domeinmodel van de webservice en de mobiele applicatie overgenomen. Hierna worden de controllers
geïmplementeerd met authenticatie en authorisatie aan de hand van de Active Directory.
Dit om vervolgens het responsive design van de webpagina's af te werken.

\section{Vergelijken van de tijd die nodig is om beide vormen van toepassingen te ontwikkelen}
Wanneer beide vormen van de toepassing ontwikkeld zijn, zal er gekeken worden naar tijd die nodig was om deze tot stand te brengen.
Dit is overigens een onderzoeksvraag van de casus en zal gebeuren aan de hand van een versiebeheersysteem.
Hierbij de tijd tussen het verschil tussen de eerste en de laatste commit zal berekend worden.

\section{Meten van gegevensverbruik}
Het meten van de verbruikte hoeveelheid gegevens zal gebeuren door te bekijken van de Debugger tijdens het uitvoeren van beide
applicaties. Dit zal een vooraf vast aantal keer gebeuren, hetzij eerst zonder en erna met caching van informatie.
Nadien zal dit proces herhaalt worden bij de responsive website. Indien alle nodige gegevens verzamelt zijn, zal er een conclusie
getrokken worden over welke vorm van applicatie het meest zuinig omgaat met het internetgebruik.

\section{Meten van gebruiksvriendelijkheid}
Nadat het gegevensverbruik gemeten is, zal ook de gebruiksvriendelijkheid gemeten worden aan de hand van enkele vragen.


\section{}
