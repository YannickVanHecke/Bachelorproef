%%=============================================================================
%% Methodologie
%%=============================================================================

\chapter{Methodologie}
\label{ch:methodologie}

%% TODO: Hoe ben je te werk gegaan? Verdeel je onderzoek in grote fasen, en
%% licht in elke fase toe welke stappen je gevolgd hebt. Verantwoord waarom je
%% op deze manier te werk gegaan bent. Je moet kunnen aantonen dat je de best
%% mogelijke manier toegepast hebt om een antwoord te vinden op de
%% onderzoeksvraag.

\section{Literatuuronderzoek}
Ter voorbereiding van deze vergelijkende studie werd er een literatuuronderzoek verricht.
Dit hield in dat er op internet gezocht werd naar verschillende wetenschappelijke artikelen over voorgaande vergelijkende studies
tussen cross platforme mobiele applicatie en responsive website. Dit literatuuronderzoek heeft al voornaamste reden het geven van
een stand van zaken in het ontwikkelen van cross-platforme mobiele applicaties en responsive webapplicatie.

\section{Vastleggen casus}
De eerste stap in het onderzoek naar de vergelijkende studie tussen een cross-platforme mobiele applicatie en een
responsive website, was het vastleggen van de de casus. Dit bepaalde de functionele requirements van beide applicaties en
gebeurde in overleg met de co-promotor. De casus omvatte een toepassing rond personeelsgegevens binnen de gentse politie,
waarbij de gegevens momenteel enkel intern beschikbaar zijn. Het doel van de casus is om deze, mits de nodige beveiliging, beschikbaar
te stellen van het personeel dat niet altijd met het intern netwerk verbonden is. Hierbij word er vooral gedacht aan mensen op het terrein.

\section{Framework kiezen voor cross platform mobiele applicatie}
Omdat vanuit de stage reeds gekozen werd voor de ontwikkelomgeving van Microsoft en dit hierdoor reeds een vertrouwde
manier van werken was, heb ik gekozen om de cross platform mobiele applicatie te ontwikkelen met Visual Studio 2015.
Hierbij wordt er gebruik gemaakt van een apart deelproject binnen de applicatie waarin code gemeenschappelijk wordt gemaakt voor
de 3 platform-specifieke applicaties. Op deze manier kan men code hergebruiken, waardoor de efficiëntie van de geschreven code stijgt.

\section{Framework kiezen voor responsive webapplicatie}
Het framework, dat voor de responsive webapplicatie gebruikt werd, is het MVC-framework van Microsoft.
Hierdoor kan de ontwikkeling van de webapplicatie ook in Visual Studio gebeuren, hetgeen voor de vergelijking van de ontwikkeltijd
een eerlijker resultaat opleverde, dan wanneer men zich nog dient in te werken in nieuwe materie.

\section{Opzetten REST api}
Het opzetten van de REST api was de volgende stap in het onderzoeksproces. Aanvankelijk werd er gekozen om eerst de basisfunctionaliteiten te implementeren en
pas nadien werd de authenticatie toegevoegd. Dit zorgt ervoor dat de data niet ongeoorloofd beschikbaar is voor derden.
Hierbij dient men zich eerst te authenticeren aan de hand van de gebruikersnaam en het wachtwoord van het gebruikersaccount binnen de Politiezone Gent.
Indien men zich succesvol geauthenticeerd heeft bij het domein aan de hand van de Active Directory, krijgt men van de REST api een token terug. De geldigheid van deze token kan men vrij kiezen.
Deze token is verplicht bij te voegen bij alle requests die verstuurd worden naar de server.
Het testen van de REST api zal eerst gebeuren aan POSTMAN, een REST-client. Op deze manier is men zeker dat de REST-api correct werkt, alvorens over te gaan te het ontwikkelen van de mobiele applicatie.

\section{Ontwikkelen cross platforme mobiele applicatie}
Nadat het testen van de REST api voltooid is, begint de ontwikkeling van de cross platforme applicatie. Hierbij is het de bedoeling om

\section{Ontwikkelen responsive webapplicatie}

\section{Vergelijken van de tijd die nodig is om beide vormen van toepassingen te ontwikkelen}

\section{Meten van gegevensverbruik bij cross platforme mobiele applicatie}

\section{Meten van gegevensverbruik bij responsive webapplicatie}
