%%=============================================================================
%% Methodologie
%%=============================================================================

\chapter{Methodologie}
\label{ch:methodologie}

%% TODO: Hoe ben je te werk gegaan? Verdeel je onderzoek in grote fasen, en
%% licht in elke fase toe welke stappen je gevolgd hebt. Verantwoord waarom je
%% op deze manier te werk gegaan bent. Je moet kunnen aantonen dat je de best
%% mogelijke manier toegepast hebt om een antwoord te vinden op de
%% onderzoeksvraag.

\section{Literatuuronderzoek}
Ter voorbereiding van deze vergelijkende studie is er een literatuuronderzoek verricht.
Hierbij werd internet gezocht naar verschillende wetenschappelijke artikelen over voorgaande vergelijkende studies
tussen cross platforme mobiele applicatie en responsive website. Dit literatuuronderzoek biedt
een stand van zaken in het ontwikkelen van cross-platforme mobiele applicaties en responsive webapplicaties.

\section{Vastleggen casus}
De eerste stap in het onderzoek naar de vergelijkende studie tussen een cross-platforme mobiele applicatie en een
responsive website, was het vastleggen van de casus in overleg met de co-promotor. Hierdoor werden de functionele requirements van beide applicaties duidelijk.
De casus omvatte een toepassing rond personeelsgegevens binnen de Gentse politie,
waarbij de gegevens momenteel enkel intern beschikbaar zijn. Het doel van de casus is om deze, mits de nodige beveiliging, beschikbaar
te stellen van het personeel dat niet altijd met het intern netwerk verbonden is. Hierbij wordt er vooral gedacht aan operationele mensen.

\section{Framework kiezen voor cross platform mobiele applicatie}
Omdat vanuit de stage reeds gekozen werd voor de ontwikkelomgeving van Microsoft en dit hierdoor reeds een vertrouwde
manier van werken was, was de keuze om de cross platform mobiele applicatie te ontwikkelen met Visual Studio 2015 een evidentie.
Hierbij wordt er gebruik gemaakt van een apart deelproject binnen de applicatie waarin code gemeenschappelijk wordt gemaakt voor
de 3 platform-specifieke applicaties. Op deze manier kan men code hergebruiken, waardoor de efficiëntie van de geschreven code stijgt.

\section{Framework kiezen voor responsive webapplicatie}
Het framework, dat voor de responsive webapplicatie gebruikt werd, is het ASP MVC 5-framework van Microsoft.
Hierdoor kan de ontwikkeling van de webapplicatie ook in Visual Studio gebeuren. Dit zorgt voor een eerlijker resultaat voor de vergelijking van de ontwikkeltijd,
dan wanneer men zich nog dient in te werken in een nieuwe technologie en ontwikkelomgeving.

\section{Opzetten REST API}
Het opzetten van de REST API was de volgende stap in het onderzoeksproces. Aanvankelijk werd ervoor gekozen om eerst de basisfunctionaliteiten te implementeren en
pas nadien de authenticatie toe te voegen. Dit zorgt ervoor dat de data niet ongeoorloofd beschikbaar is voor derden.
Hierbij dient men zich eerst te authenticeren aan de hand van de gebruikersnaam en het wachtwoord van het gebruikersaccount binnen de Politiezone Gent.
Indien men zich succesvol geauthenticeerd heeft bij het domein aan de hand van de Active Directory, krijgt men van de REST API een token terug. De geldigheid van deze token kan men vrij kiezen.
Deze token is verplicht bij te voegen bij alle requests die verstuurd worden naar de server.
Het testen van de REST API zal eerst gebeuren aan POSTMAN, een REST-client. Zo is men zeker dat de REST API correct werkt, alvorens over te gaan te het ontwikkelen van de mobiele applicatie.

\section{De ontwikkeling van een cross platforme mobiele applicatie}
Nadat het testen van de REST API voltooid is, begint de ontwikkeling van de cross platforme applicatie.
Het is hierbij de bedoeling om code gemeenschappelijk te gebruiken in de drie deelprojecten. Concreet zal dat de code zijn om gegevens
uit de webservice op te halen, samen met de code om deze op te slaan in een lokale databank.

 Op deze manier wordt de ontwikkeltijd gereduceerd ten opzichte van het ontwikkelen van 3 afzonderlijke mobiele applicaties.
Het weergeven van de opgehaalde data staat, samen met het ontwerp van de schermen voor de platforme-specifieke user interfaces,
 gecodeerd in de platform-specifieke projecten. Dit omdat er geen uniforme manier bestaat om de data weer te geven voor de 3 besturingssystemen binnen
de gekozen ontwikkelomgeving.

\section{De ontwikkeling van een responsive webapplicatie}
De volgende stap is het ontwikkelen van de responsive webapplicatie.
Hierbij wordt dezelfde databank gebruikt van de webservice van de cross platforme
mobiele applicatie. Verder wordt ook het domeinmodel van de webservice en de mobiele applicatie overgenomen. Hierna worden de controllers
geïmplementeerd met authenticatie en authorisatie aan de hand van de REST API.
Dit om vervolgens het responsive design van de webpagina's af te werken.

\section{Vergelijken van de tijd die nodig is om beide vormen van toepassingen te ontwikkelen}
Wanneer beide vormen van de toepassing ontwikkeld zijn, zal er gekeken worden naar tijd die nodig was om deze tot stand te brengen.
Dit is overigens een onderzoeksvraag van de casus en zal gebeuren aan de hand van een versiebeheersysteem.
Hierbij zal het verschil in tijd tussen de eerste en de laatste commit van ieder project berekend worden.

\section{Meten van het gegevensverbruik}
De verbruikte hoeveelheid gegevens wordt gemeten door de grootte van het HttpResponse-object uit te lezen.
Dit HttpResponse-object is het antwoord op de aanvraag voor gegevens bij de REST API. Vooraf wordt bepaald hoeveel keer dit gebeurt.
Nadien wordt dit proces herhaald bij de responsive website. Indien alle nodige gegevens verzameld zijn, wordt een conclusie
getrokken over welke vorm van applicatie het meest zuinig omgaat met het internetgebruik.

\section{Meten van de snelheid}
De snelheid van beide vormen van applicatie is een criterium waarop men beoordeelt tussen de keuze tussen de app en de website.
Men wil namelijk zo snel mogelijk de gevraagde gegevens ter beschikking hebben. De ingebouwde stopwatch in het .NET-framework meet
dit. Nadien worden deze resultaten meegenomen in de besluitvorming.

\section{Extra kosten bij de realisatie en publicatie}
Om te kiezen voor de cross-platforme mobiele applicatie of de responsive website, dient rekening gehouden te worden met de kosten die nodig zijn voor de realisatie en publicatie.
Hierbij wordt hoofdzakelijk gedacht aan de kosten die de publicatie in de app stores met zich meebrengt, alsook de kosten voor de hosting van enerzijds de webservice en
anderzijds de webapplicatie. Deze kosten zijn online beschikbaar.

\section{Besluitvorming}
Eens alle bovenvermelde gegevens bekend zijn, worden deze met elkaar vergeleken om zo tot een gefundeerde lijst van argumenten
te komen. Deze lijst vermeldt in welke gevallen de cross-platforme mobiele applicatie een betere oplossing kan zijn voor de klant.
Overigens wordt er een lijst met karakteristieken voor de responsive webapplicatie bekend gemaakt.
