%%=============================================================================
%% Methodologie
%%=============================================================================

\chapter{Methodologie}
\label{ch:methodologie}

%% TODO: Hoe ben je te werk gegaan? Verdeel je onderzoek in grote fasen, en
%% licht in elke fase toe welke stappen je gevolgd hebt. Verantwoord waarom je
%% op deze manier te werk gegaan bent. Je moet kunnen aantonen dat je de best
%% mogelijke manier toegepast hebt om een antwoord te vinden op de
%% onderzoeksvraag.

\section{literatuuronderzoek}
Ter voorbereiding van het onderzoek 

\section{vastleggen casus}
De eerste stap in het onderzoek naar de vergelijkende studie tussen een cross-platforme mobiele applicatie en een
responsive website, is het vastleggen van de de casus. Dit bepaalt de functionele requirements van beide applicaties en
gebeurde in overleg met de co-promotor.
\section{Framework kiezen voor cross platform mobiele applicatie}
Omdat vanuit de stage reeds gekozen werd voor de ontwikkelomgeving van Microsoft en hierdoor er reeds vooraf een vertrouwde
manier van werken was, heb ik gekozen om de cross platform mobiele applicatie te ontwikkelen met Visual Studio 2015.
Hierbij wordt er gebruik gemaakt van een apart deelproject binnen de applicatie waarin code gemeenschappelijk wordt gemaakt voor
de 3 platform-specifieke applicaties. Op deze manier kan men code hergebruiken, waardoor de efficiëntie van de geschreven code stijgt.


\section{Framework kiezen voor responsive webapplicatie}

\section{Opzetten REST api}


\section{Ontwikkelen cross platforme mobiele applicatie}
\section{Ontwikkelen responsive webapplicatie}
\section{Meten van gegevensverbruik bij cross platforme mobiele applicatie}
\section{Meten van gegevensverbruik bij responsive webapplicatie}
\lipsum[21-25]
